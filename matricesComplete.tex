\documentclass{article}
\usepackage[utf8]{inputenc}
\usepackage{amsmath}
\usepackage{amsfonts}
\usepackage{mathtools}

\usepackage{xcolor}
\usepackage{pifont}

\title{Mathematical foundations for Howard algorithm}
\author{Elias Heikkilä, Mahdi Baianifar}
\date{}
\setlength{\parindent}{0em}
\begin{document}
	\maketitle
	\subsection*{Basic definitions}
	Define $2 \times 2$ Pauli matrices: $\mathbf{I}_2, \mathbf{X} = \begin{bmatrix} 0 && 1 \\ 1 && 0 \end{bmatrix}, \mathbf{Z} = \begin{bmatrix} 1 && 0 \\ 0 && -1 \end{bmatrix}, \mathbf{Y} = i\mathbf{X}\mathbf{Z} = \begin{bmatrix} 0 && -i \\ i && 0 \end{bmatrix}$. Pauli matrices are Hermitian.
	Let $m \in \mathbb{N}$ and $N = 2^m$. Define a $N \times N$ D-matrix with binary vectors $\mathbf{a} = (a_1, ..., a_m), \mathbf{b} = (b_1,..., b_m) \in \mathbb{F}_2^m$ as follows:
	\begin{align*}
		\mathbf{D}(\mathbf{a}, \mathbf{b}) = \mathbf{X}^{a_1}\mathbf{Z}^{b_1} \otimes \cdots \otimes \mathbf{X}^{a_m}\mathbf{Z}^{b_m}.
	\end{align*}
	Now $N \times N$ Pauli matrix is defined as $\mathbf{E}(\mathbf{a}, \mathbf{b}) = i^{\mathbf{a}\mathbf{b}^T}\mathbf{D}(\mathbf{a}, \mathbf{b})$. It is clearly Hermitian as there are imaginary units for every $\mathbf{XZ}$ term, so they can be replaced by $Y$ matrices. The tensor product is Hermitian as every element in the product is Hermitian.
	\subsection*{Hadamard matrices}
	Denote by $\mathbf{H}_2 = \dfrac{1}{\sqrt{2}}\begin{bmatrix} 1 & 1 \\ -1 & 1 \end{bmatrix}$ the $2 \times 2$ Walsh-Hadamard matrix. Define $N\times N$ Walsh-Hadamard matrix as a tensor product $\mathbf{H}_N = \mathbf{H}^{\otimes m}_2$. Now we can notice that $\mathbf{H}_2 = \frac{1}{\sqrt{2}}(\mathbf{I} - \mathbf{XZ})$ and by distributivity, the tensor product gets forms
	\begin{equation}\label{HadamardNonNat}
		\mathbf{H}_N = \bigotimes_{1=1}^m \mathbf{H}_2 =	\bigotimes_{i = 1}^m \frac{1}{\sqrt{2}} (\mathbf{I} - \mathbf{XZ}) = \frac{1}{\sqrt{N}}\sum_{\mathbf{a} \in \mathbb{F}_2^m} (-1)^{w(\mathbf{a})}\mathbf{D}(\mathbf{a},\mathbf{a}) = \frac{1}{\sqrt{N}} \sum_{\mathbf{a} \in \mathbb{F}_2^m} \mathbf{D}(\mathbf{0},\mathbf{a})\mathbf{D}(\mathbf{a},\mathbf{0}).
	\end{equation}
	We can also continue calculations to give more equivalent forms
	\begin{align*}
		\frac{1}{\sqrt{N}}	\sum_{\mathbf{a} \in \mathbb{F}_2^m} (-1)^{w(\mathbf{a})}\mathbf{D}(\mathbf{a}, \mathbf{a}) = \frac{1}{\sqrt{N}}\sum_{\mathbf{a} \in \mathbb{F}_2^m} i^{w(\mathbf{a})}\mathbf{E}(\mathbf{a}, \mathbf{a}) = \frac{1}{\sqrt{N}} \prod_{i = 1}^m (\mathbf{I} + i\mathbf{E}(\mathbf{e}_i, \mathbf{e}_i)) = \frac{1}{\sqrt{N}} \prod_{i = 1}^m (\mathbf{I} - \mathbf{D}(\mathbf{e}_i, \mathbf{e}_i))
	\end{align*}
	
	
	% TODO get rid of the signs in sum of D matrices and use the Hadamard matrix of the form \sum D(a,a).
	
	% TODO Find a reference or write a proof that binary symmetric matrix can be written as a sum of at most m + 1 transvections.
	% Also it would be great to find explicitly how the CLifford g = diag(i^xSx) is decomposed into transvections.
	
	This is the version of Hadamard matrix where the diagonal is all ones. We can have the "naturally" ordered Hadamard matrix by considering a product $\begin{bmatrix} 1 & 1 \\ 1 & -1 \end{bmatrix}^{\otimes m}$. As $\begin{bmatrix} 1 & 1 \\ 1 & -1 \end{bmatrix} = \begin{bmatrix} 1 & 1 \\ -1 & 1 \end{bmatrix} \begin{bmatrix} 0 & 1 \\ 1 & 0 \end{bmatrix}$, we can switch between the different versions by multiplying $\mathbf{H}_N\mathbf{X}^{\otimes m}$ i.e. reversing the order of the columns.
	There is a neat sum form for the naturally ordered Hadamard:
	\begin{align*}
		\mathbf{H}^{\text{nat}}_N = \frac{1}{\sqrt{N}} \sum_{\mathbf{a} \in \mathbb{F}_2^m} \mathbf{D}(\mathbf{0}, \mathbf{a}) \mathbf{D}(\overline{\mathbf{a}}, \mathbf{0})
	\end{align*}
	where $\overline{\mathbf{a}}$ is a bitwise complement of $\mathbf{a}$. We will denote the "all ones diagonal" version of $N \times N$ Walsh-Hadamard matrix simply by $\mathbf{H}$. 
	Walsh-Hadamard matrix belongs to the Clifford group as it permutes the Pauli group. Walsh-Hadamard matrix corresponds to the symplectic matrix $\mathbf{F} = \begin{bmatrix}
		\mathbf{0} & \mathbf{I} \\
		\mathbf{I} & \mathbf{0}
	\end{bmatrix}$
	so this means that if we conjugate $\mathbf{E}(\mathbf{a},\mathbf{b})$ with the Walsh-Hadamard matrix $\mathbf{H}$, we will swap the places of $\mathbf{a}$ and $\mathbf{b}$: $\mathbf{H}^H\mathbf{E}(\mathbf{a},\mathbf{b})\mathbf{H} = \pm \mathbf{E}(\left[\mathbf{a},\mathbf{b}\right]\mathbf{F}) = \pm \mathbf{E}(\mathbf{b},\mathbf{a})$. Now it is clear that Walsh-Hadamard matrix gives an isomorphism between the $\mathcal{X}$-group: $\mathcal{X} = \{\mathbf{E}(\mathbf{a},\mathbf{0}) \mid \mathbf{a} \in \mathbb{F}^m_2 \}$ and $\mathcal{Z}$-group: $\mathcal{Z} = \{\mathbf{E}(\mathbf{0},\mathbf{b}) \mid \mathbf{b} \in \mathbb{F}^m_2 \}$ since $\mathbf{H}^H\mathcal{X}\mathbf{H} = \mathcal{Z}$. The $\mathcal{X}$-group is a maximal commuting subgroup of the Pauli group.
	\\
	
	\subsection{PRELIMINARIES}
	Before considering the definition of the binary Chirps, we discuss little about some main theorems that will be used hereafter.
	
	
	 First, we notice that 
	\begin{equation}\label{Xtypes}
		D\left(\mathbf{a}, \mathbf{0}\right) = \sum_{\mathbf{v} \in \mathbb{F}_2^m}{|\mathbf{v+a}><\mathbf{v}|}
	\end{equation} 
To see this fact, first we notice that $\mathbf{x} = |0><1| + |1><0|$ and also $\mathbf{I} = |0><0| + |1><1|$, consider the fact that 
\begin{equation*}
	\mathbf{x}^{a_1}  = (1-a_1)\bigg{(}|0><0| + |1><1|\bigg{)}+a_1 \bigg{(} |0><1| + |1><0| \bigg{)}
\end{equation*}
Then, we have
\begin{align*}
	\mathbf{x}^{a_1} \otimes \mathbf{x}^{a_2} =& (1-a_1)(1-a_2)\bigg{(} |00><00|+|01><01|+|10><10|+|11><11| \bigg{)}\\
	&+ (1-a_1)a_2\bigg{(} |00><01|+|01><00|+|10><11|+|11><10| \bigg{)} \\
	&+a_1(1-a_2)\bigg{(} |00><10|+|01><11|+|10><00|+|11><01| \bigg{)} \\
	&+a_1 a_2 \bigg{(} |00><11|+|01><10|+|10><01|+|11><00| \bigg{)}
\end{align*}
Then it can be seen that $x$ shifts the basis vectors according to the vector $\mathbf{a}$. Hence it can be seen that Eq. \eqref{Xtypes} is correct.


Using similar approach, we can see that 
\begin{equation}\label{Ztypes}
	D\left(\mathbf{0},\mathbf{a}\right) = \sum_{\mathbf{v}\in \mathbb{F}_2^m}{ (-1)^{\mathbf{a} \mathbf{v}^T} |\mathbf{v}><\mathbf{v}|}
\end{equation}
Then combining Eq. \eqref{Xtypes} and \eqref{Ztypes}, we have 
\begin{align}\label{XZtypes}
	D\left(\mathbf{a}, \mathbf{b}\right) &= D\left(\mathbf{a}, \mathbf{0}\right) D\left(\mathbf{0}, \mathbf{b}\right) = \sum_{\mathbf{v} \in \mathbb{F}_2^m}{|\mathbf{v+a}><\mathbf{v}|}  \sum_{\mathbf{v}'\in \mathbb{F}_2^m}{ (-1)^{\mathbf{v}' \mathbf{b}^T} |\mathbf{v}'><\mathbf{v}'|} \nonumber \\
	& = \sum_{\mathbf{v} \in \mathbb{F}_2^m}{(-1)^{\mathbf{v} \mathbf{b}^T} |\mathbf{v+a}><\mathbf{v}|}
\end{align}

	Also, we can rewrite $\mathbf{H}_N^{\text{nat}}$ as follows
\begin{equation}\label{Hrefomed}
	\mathbf{H}_N^{\text{nat}} = \frac{1}{\sqrt{N}} \sum_{\mathbf{u}, \mathbf{v} \in \mathbf{F}_2^m}{\left(-1\right)^{\mathbf{u} \mathbf{v}^T } |\mathbf{u}><\mathbf{v}|}
\end{equation}
	But how? First we start with the not-natural definition of the Hadamard matrix i.e., Eq. \ref{HadamardNonNat} and considering  $\mathbf{H}_N = \frac{1}{\sqrt{N}} \sum_{\mathbf{a} \in \mathbb{F}_2^m} (-1)^{w(\mathbf{a})} D\left( \mathbf{a}, \mathbf{a} \right)$ and substituting \eqref{XZtypes}, we get
	\begin{equation}\label{HnotNat}
		\mathbf{H}_N = \frac{1}{\sqrt{N}} \sum_{\mathbf{u}, \mathbf{v} \in \mathbf{F}_2^m}{\left(-1\right)^{\mathbf{u} (\mathbf{v+u})^T } |\mathbf{u}><\mathbf{v}|}
	\end{equation}
	that differs from	Eq. \eqref{Hrefomed}!! Since \eqref{HnotNat} is related with non-natural Hadamard gate, however, Eq. \eqref{Hrefomed} is used for natural Hadamard gate. Relation between them? We notice that $\mathbf{H}_N^{\text{nat}} = \mathbf{H}\mathbf{X}^{\otimes m}$, using \eqref{HnotNat}, we get
	\begin{align*}
		&\frac{1}{\sqrt{N}} \sum_{\mathbf{u}, \mathbf{v} \in \mathbf{F}_2^m}{\left(-1\right)^{\mathbf{u} (\mathbf{v+u})^T } |\mathbf{u}><\mathbf{v}|}\mathbf{D}(\mathbf{1,0}) = \\
		& \frac{1}{\sqrt{N}} \sum_{\mathbf{u}, \mathbf{v} \in \mathbf{F}_2^m}{\left(-1\right)^{\mathbf{u} (\mathbf{v+u})^T } |\mathbf{u}><\mathbf{v}|}\sum_{\mathbf{x} \in \mathbf{F}_2^m}{ |\mathbf{x+1}><\mathbf{x}|} \\
		& =  \frac{1}{\sqrt{N}} \sum_{\mathbf{u}, \mathbf{v} \in \mathbf{F}_2^m}{\left(-1\right)^{\mathbf{u} (\mathbf{v+u+1})^T } |\mathbf{u}><\mathbf{v}|}
	\end{align*}
	where $\mathbf{u}(\mathbf{v+u+1})^T=\mathbf{u}(\mathbf{v}^T$ since $\mathbf{u}(\mathbf{u+1})^T=0$, and hence, we get the \eqref{Hrefomed}.
	
	Hereafter, we try to show that column of Hadamard matrix are eigenvectors of $D\left(\mathbf{a,0}\right)$. We can see that, $i$th column of $\mathbf{H}_N$ can be written as 
	\begin{equation}\label{ithColHn}
		\mathbf{H}_N^{\text{nat}} \mathbf{e}_i = \mathbf{H}_N^{\text{nat}} | \mathbf{v}_{\mathbf{e}_i}> = \frac{1}{\sqrt{N}} \sum_{\mathbf{u} \in \mathbf{F}_2^m}{\left(-1\right)^{\mathbf{u} \mathbf{v}_{\mathbf{e}_i}^T } |\mathbf{u}>}
	\end{equation}
	Considering Eq. \eqref{Xtypes} and \eqref{ithColHn}, we have
\begin{align*}
	D\left(\mathbf{a}, \mathbf{0} \right) \mathbf{H}_N^{\text{nat}} | \mathbf{v}_{\mathbf{e}_i} > \: & = \sum_{\mathbf{v} \in \mathbb{F}_2^m}{|\mathbf{v+a}><\mathbf{v}|}  \frac{1}{\sqrt{N}} \sum_{\mathbf{u} \in \mathbf{F}_2^m}{\left(-1\right)^{\mathbf{u} \mathbf{v}_{\mathbf{e}_i}^T } |\mathbf{u} > } \\
	& = \frac{1}{\sqrt{N}} \sum_{\mathbf{v} \in \mathbb{F}_2^m}{(-1)^{\mathbf{v}\mathbf{v}_{\mathbf{e}_i}^T } |\mathbf{v+a}>} \\
	& = \frac{1}{\sqrt{N}} \sum_{\mathbf{v} \in \mathbb{F}_2^m}{(-1)^{\left(\mathbf{v+a}\right)\mathbf{v}_{\mathbf{e}_i}^T } |\mathbf{v}>}\\
	& = (-1)^{\mathbf{a}\mathbf{v}_{\mathbf{e}_i}^T} \mathbf{H}_N^{\text{nat}} \mathbf{e}_i
\end{align*} 
		Also, we can use other approach as follows
	\begin{align*}
		D\left(\mathbf{b},\mathbf{0}\right) \mathbf{H}_N |v>& \: = \frac{1}{\sqrt{N}}\sum_{\mathbf{a} \in \mathbb{F}_2^m}{D\left(\mathbf{b},\mathbf{0}\right) D\left(\mathbf{0},\mathbf{a}\right) D\left(\mathbf{a},\mathbf{0}\right)}|v> = \frac{1}{\sqrt{N}}\sum_{\mathbf{a} \in \mathbb{F}_2^m}{(-1)^{\mathbf{a}\mathbf{a}^T} D\left(\mathbf{b}+\mathbf{a},\mathbf{a}\right)}|v> \\
		& = \frac{1}{\sqrt{N}}\sum_{\mathbf{a} \in \mathbb{F}_2^m}{(-1)^{(\mathbf{a+v})\mathbf{a}^T} | \mathbf{v}+\mathbf{a}+\mathbf{b}>}=\frac{1}{\sqrt{N}}\sum_{\mathbf{x} \in \mathbb{F}_2^m}{(-1)^{(\mathbf{b+x+v})(\mathbf{x+b})^T} | \mathbf{v}+\mathbf{x}>}= \\
		&\frac{(-1)^{(\mathbf{v+b})
				\mathbf{b}^T }}{\sqrt{N}}\sum_{\mathbf{x} \in \mathbb{F}_2^m}{(-1)^{(\mathbf{v+x})\mathbf{x}^T} | \mathbf{v}+\mathbf{x}>}=\frac{(-1)^{(\mathbf{v+b})
				\mathbf{b}^T }}{\sqrt{N}}\sum_{\mathbf{x} \in \mathbb{F}_2^m}{D(\mathbf{0, x}) D(\mathbf{x, 0})}|v>
	\end{align*}
	Where the last term is equal to $\mathbf{H}_N | \mathbf{v}>$. We note that multiplying the result with $\mathbf{X}^{\otimes m}$, we get similar result as in \eqref{ithColHn}.
	\subsection*{Binary chirp codebook}
	Now the symplectic matrix $\mathbf{T}_\mathbf{S} = \begin{bmatrix} \mathbf{I} & \mathbf{S} \\ \mathbf{0} & \mathbf{I} \end{bmatrix}$, where $\mathbf{S}$ is a binary symmetric matrix, corresponds to the Clifford operator $\mathbf{g} = \text{diag}(i^{\mathbf{v}\mathbf{S}\mathbf{v}^T})$. (This is where the generalization happens when we consider arbitrary binary matrices or matrices with half-elements in the generalized setting). \\	 
	
	When we conjugate Walsh-Hadamard matrix with $\mathbf{g}$, we get a matrix of codewords determined by $\mathbf{S}$. We can use two different approach. 
	
	\begin{enumerate}
		\item First case:
		$\mathbf{W=gH}$. In this case the codewords are the columns (or rows) of the following matrix:
		
		\begin{align}
			\mathbf{W} &= \mathbf{g}\mathbf{H} = \frac{1}{\sqrt{N}}\mathbf{g}\left(\sum_{\mathbf{a} \in \mathbb{F}^m_2} i^{w(\mathbf{a})}\mathbf{E}(\mathbf{a},\mathbf{a})\right) = \frac{1}{\sqrt{N}}\sum_{\mathbf{a} \in \mathbb{F}^m_2} i^{w(\mathbf{a})}\mathbf{g}\mathbf{E}(\mathbf{a},\mathbf{a}) \label{WfirstDef} \\
			\overline{\mathbf{W}} &= \frac{1}{\sqrt{N}}\sum_{\mathbf{a} \in \mathbb{F}^m_2} i^{-w(\mathbf{a})}(-1)^{w(\mathbf{a})}\overline{\mathbf{g}}\mathbf{E}(\mathbf{a},\mathbf{a}) = \frac{1}{\sqrt{N}}\sum_{\mathbf{a} \in \mathbb{F}^m_2} i^{w(\mathbf{a})}\overline{\mathbf{g}}\mathbf{E}(\mathbf{a},\mathbf{a}) \label{WbarFirstDef}
		\end{align}
	\item Second case: $\mathbf{W=gHg}^H$. In this case we have
	\begin{align}
		\mathbf{W} &= \mathbf{g}\mathbf{H}\mathbf{g}^H = \frac{1}{\sqrt{N}}\mathbf{g}^H\left(\sum_{\mathbf{a} \in \mathbb{F}^m_2} i^{w(\mathbf{a})}\mathbf{E}(\mathbf{a},\mathbf{a})\right)\mathbf{g} = \frac{1}{\sqrt{N}}\sum_{\mathbf{a} \in \mathbb{F}^m_2} i^{w(\mathbf{a})}\mathbf{g}\mathbf{E}(\mathbf{a},\mathbf{a})\mathbf{g}^H\nonumber  \\ 
		&=\frac{1}{\sqrt{N}}\sum_{\mathbf{a} \in \mathbb{F}^m_2} i^{w(\mathbf{a})}\mathbf{E}(\mathbf{a, a+aS}) \label{WsecDef}
	\end{align}
where the last equality come from the fact that $\mathbf{g}\mathbf{E}(\mathbf{a},\mathbf{a}) \mathbf{g}^H = +\mathbf{E}(\mathbf{a,a+a S})$. Here we can prove it as follows
	\begin{align*}
		\mathbf{g} \mathbf{E}(\mathbf{a,a}) \mathbf{g}^H &=\sum_{\mathbf{v} \in \mathbb{F}^m_2}{i^{\mathbf{v S v^T}} |\mathbf{v}><\mathbf{v}|} i^{\mathbf{a}\mathbf{a}^T}\sum_{\mathbf{u} \in \mathbb{F}^m_2}{(-1)^{\mathbf{a}\mathbf{u}^T} |\mathbf{u+a}><\mathbf{u}|}\sum_{\mathbf{z} \in \mathbb{F}^m_2}{i^{-\mathbf{z S z^T}} |\mathbf{z}><\mathbf{z}|} \nonumber \\
		&= i^{\mathbf{a}\mathbf{a}^T}\sum_{\mathbf{u} \in \mathbb{F}^m_2}{i^{\mathbf{(u+a)S}(\mathbf{u+a})^T} (-1)^{\mathbf{a}\mathbf{u}^T}i^{-\mathbf{uS}\mathbf{u}^T} |\mathbf{u+a}><\mathbf{u}|} \nonumber \\
		&= i^{\mathbf{a+aS}\mathbf{a}^T}\sum_{\mathbf{u} \in \mathbb{F}^m_2}{(-1)^{(\mathbf{a+aS})\mathbf{u}^T} |\mathbf{u+a}><\mathbf{u}|}= i^{(\mathbf{a+aS})\mathbf{a}^T}\mathbf{D}(\mathbf{a, a+aS}) = \mathbf{E}(\mathbf{a, a+aS})
	\end{align*}
Also, from Eq. \ref{WsecDef}, we can see that element-wise conjugate or $\mathbf{W}$, can be written as
\begin{equation}\label{WbarSecDef}
	\overline{\mathbf{W}} = \frac{1}{\sqrt{N}} \sum_a i^{-w(\mathbf{a})}(-1)^{\mathbf{a}(\mathbf{a}\mathbf{S} + \mathbf{a})^T} E(\mathbf{a}, \mathbf{a}(\mathbf{S} + \mathbf{I}))
\end{equation}
	\end{enumerate}
In both cases $\overline{\mathbf{W}}$ denotes the elementwise complex conjugate of $\mathbf{W}$.

	\subsection*{Howard algorithm}
	Now calculate a "shift" of the codeword matrix according to the definition of the two aforementioned cases.
	\begin{enumerate}
		\item First case: $\mathbf{W=gH}$. Considering Eq. \ref{WfirstDef}, we have
		\begin{align*}
			\mathbf{E}(\mathbf{e}_j, \mathbf{0})\mathbf{W} = &\frac{1}{\sqrt{N}} \sum_{\mathbf{a} \in \mathbb{F}^m_2} i^{w(\mathbf{a})}\mathbf{E}(\mathbf{e}_j,\mathbf{0})\mathbf{g}\mathbf{E}(\mathbf{a},\mathbf{a}) 
		\end{align*}
		We must calculate $\mathbf{E}(\mathbf{e}_j,\mathbf{0})\mathbf{g}\mathbf{E}(\mathbf{a},\mathbf{a})$. We have
		\begin{align*}
			\mathbf{E}(\mathbf{e}_j,\mathbf{0})\mathbf{g}\mathbf{E}(\mathbf{a},\mathbf{a})& = 	i^{w(\mathbf{a})}  \sum_{\mathbf{v} \in \mathbb{F}_2^m}{ |\mathbf{v}+\mathbf{e}_j><\mathbf{v}|}  \sum_{\mathbf{u} \in \mathbb{F}_2^m}{i^{\mathbf{u S}\mathbf{u}^T} |\mathbf{u}><\mathbf{u}|}\sum_{\mathbf{x} \in \mathbb{F}_2^m}{ (-1)^{\mathbf{x}\mathbf{a}^T}|\mathbf{x+a}><\mathbf{x}|} \\
			\\
			&= i^{w(\mathbf{a})}  \sum_{\mathbf{v} \in \mathbb{F}_2^m}{ |\mathbf{v}+\mathbf{e}_j><\mathbf{v}|}  \sum_{\mathbf{u} \in \mathbb{F}_2^m}{i^{\mathbf{u S}\mathbf{u}^T} |\mathbf{u}><\mathbf{u}|} (-1)^{(\mathbf{u+a})\mathbf{a}^T}|\mathbf{u}><\mathbf{u+a}| \\
			&= i^{w(\mathbf{a})} \sum_{\mathbf{v} \in \mathbb{F}_2^m}{i^{\mathbf{v}\mathbf{S}\mathbf{v}^T}(-1)^{(\mathbf{v+a})\mathbf{a}^T} |\mathbf{v}+\mathbf{e}_j><\mathbf{v+a}|} \\
			& = i^{w(\mathbf{a})} \sum_{\mathbf{v} \in \mathbb{F}_2^m}{i^{(\mathbf{v+a})\mathbf{S}(\mathbf{v+a})^T}(-1)^{\mathbf{v}\mathbf{a}^T} |\mathbf{v+a}+\mathbf{e}_j><\mathbf{v}|}
		\end{align*}
		Then, we have
		\begin{align*}
			\mathbf{E}(\mathbf{e}_j, \mathbf{0})\mathbf{W} &= \frac{1}{\sqrt{N}}  \sum_{\mathbf{a} \in \mathbb{F}_2^m}{\sum_{\mathbf{v} \in \mathbb{F}_2^m}{ (-1)^{w(\mathbf{a})} i^{(\mathbf{v+a})\mathbf{S}(\mathbf{v+a})^T}(-1)^{\mathbf{v}\mathbf{a}^T} |\mathbf{v+a}+\mathbf{e}_j><\mathbf{v}|}}\\
			& = \frac{1}{\sqrt{N}}  \sum_{\mathbf{a} \in \mathbb{F}_2^m}{\sum_{\mathbf{v} \in \mathbb{F}_2^m}{ (-1)^{w(\mathbf{a})} i^{\mathbf{a}\mathbf{S}\mathbf{a}^T}(-1)^{\mathbf{v}\mathbf{a}^T} |\mathbf{a}+\mathbf{e}_j><\mathbf{v}|}}
		\end{align*}
		where at the last equality, we replaced $\mathbf{a=a+v}$. 
		
		Denote by $\overline{\mathbf{W}}$ the elementwise complex conjugate of the matrix $\mathbf{W}$. Consider the pointwise product
		
		\begin{align*}
			\overline{\mathbf{W}} \odot \mathbf{E}(\mathbf{e}_j, 0)\mathbf{W}& = \frac{1}{N} \sum_{\mathbf{a} \in \mathbb{F}^m_2} i^{w(\mathbf{a})}\overline{\mathbf{g}}\mathbf{E}(\mathbf{a},\mathbf{a})\odot \sum_{\mathbf{b} \in \mathbb{F}_2^m}{\sum_{\mathbf{v} \in \mathbb{F}_2^m}{ (-1)^{w(\mathbf{b})} i^{\mathbf{b}\mathbf{S}\mathbf{b}^T}(-1)^{\mathbf{v}\mathbf{b}^T} |\mathbf{b}+\mathbf{e}_j><\mathbf{v}|}}\\
			&\frac{1}{N} \sum_{\mathbf{a} \in \mathbb{F}^m_2}{\sum_{\mathbf{b} \in \mathbb{F}_2^m}{\sum_{\mathbf{v} \in \mathbb{F}_2^m}}i^{w(\mathbf{a})}(-1)^{(\mathbf{b+v})\mathbf{b}^T}  i^{\mathbf{b}\mathbf{S}\mathbf{b}^T} (\overline{\mathbf{g}}\mathbf{E}(\mathbf{a},\mathbf{a}))\odot (|\mathbf{b}+\mathbf{e}_j><\mathbf{v}|) }
		\end{align*}
		Thus, we must calculate $(\overline{\mathbf{g}}\mathbf{E}(\mathbf{a},\mathbf{a}))\odot (|\mathbf{b}+\mathbf{e}_j><\mathbf{v}|)$
		\begin{align*}
			\overline{\mathbf{g}}\mathbf{E}(\mathbf{a},\mathbf{a}) &=i^{w(\mathbf{a})} \sum_{\mathbf{u} \in \mathbb{F}_2^m}{i^{-\mathbf{u S}\mathbf{u}^T} |\mathbf{u}><\mathbf{u}|}\sum_{\mathbf{x} \in \mathbb{F}_2^m}{ (-1)^{\mathbf{x}\mathbf{a}^T}|\mathbf{x+a}><\mathbf{x}|} \\
			&=i^{-w(\mathbf{a})} \sum_{\mathbf{u} \in \mathbb{F}_2^m}{(-1)^{\mathbf{u}\mathbf{a}^T}i^{-\mathbf{u S}\mathbf{u}^T} |\mathbf{u}><\mathbf{u+a}|},
		\end{align*} 
		then we get
		\begin{align*}
			&\overline{\mathbf{W}} \odot \mathbf{E}(\mathbf{e}_n, 0)\mathbf{W} = \\
			& \frac{1}{N} \sum_{\mathbf{a} \in \mathbb{F}^m_2}{\sum_{\mathbf{b} \in \mathbb{F}_2^m}{\sum_{\mathbf{v} \in \mathbb{F}_2^m}}(-1)^{(\mathbf{b+v})\mathbf{b}^T}  i^{\mathbf{b}\mathbf{S}\mathbf{b}^T} \sum_{\mathbf{u} \in \mathbb{F}_2^m}{(-1)^{\mathbf{u}\mathbf{a}^T}i^{-\mathbf{u S}\mathbf{u}^T} |\mathbf{u}><\mathbf{u+a}|} \odot (|\mathbf{b}+\mathbf{e}_j><\mathbf{v}|) }
		\end{align*}
		We notice that $|\mathbf{u}><\mathbf{u+a}| \odot (|\mathbf{b}+\mathbf{e}_j><\mathbf{v}|) = 1$ iff $\mathbf{u}=\mathbf{b}+\mathbf{e}_j, \mathbf{u+a}=\mathbf{v}$. Then, we can substitute $\mathbf{u} = \mathbf{b}+\mathbf{e}_j$ and $\mathbf{v}=\mathbf{a}+\mathbf{b}+\mathbf{e}_j$. Thus, we have
		\begin{align}
			&\frac{1}{N}\sum_{\mathbf{a} \in \mathbb{F}^m_2}{\sum_{\mathbf{b} \in \mathbb{F}^m_2}{i^{\mathbf{b S b}^T -(\mathbf{b+e}_j)\mathbf{S}(\mathbf{b+e}_j)^T} (-1)^{(2\mathbf{b}+\mathbf{a+e}_j)\mathbf{b}^T+(\mathbf{b+e}_j)\mathbf{a}^T} |\mathbf{b+e}_j> <\mathbf{a+b+e}_j |    } }\nonumber \\
			&\frac{1}{N}\sum_{\mathbf{a} \in \mathbb{F}^m_2}{\sum_{\mathbf{b} \in \mathbb{F}^m_2}{i^{-2\mathbf{b S }\mathbf{e}_j^T-\mathbf{e}_j \mathbf{S e}_j^T} (-1)^{(\mathbf{a+b})\mathbf{e}_j^T} |\mathbf{b+e}_j> <\mathbf{a+b+e}_j |    } }\nonumber\\
			& \frac{1}{N}\sum_{\mathbf{a} \in \mathbb{F}^m_2}{\sum_{\mathbf{b} \in \mathbb{F}^m_2}{i^{-2\mathbf{(a+b) S }\mathbf{e}_j^T+\mathbf{e}_j \mathbf{S e}_j^T} (-1)^{(\mathbf{b+e}_j)\mathbf{e}_j^T} |\mathbf{a+b}> <\mathbf{b} |    } }\nonumber \\
			& \frac{-1}{N}\sum_{\mathbf{a} \in \mathbb{F}^m_2}{\sum_{\mathbf{b} \in \mathbb{F}^m_2}{i^{-2\mathbf{a S }\mathbf{e}_j^T+\mathbf{e}_j \mathbf{S e}_j^T} (-1)^{(\mathbf{e}_j+\mathbf{e}_j\mathbf{S})\mathbf{b}^T} |\mathbf{b+a}> <\mathbf{b} |    } } \nonumber \\
			& \frac{-1}{N}\sum_{\mathbf{a} \in \mathbb{F}^m_2}{i^{-2\mathbf{a S }\mathbf{e}_j^T+\mathbf{e}_j \mathbf{S e}_j^T} \mathbf{D(a,e}_j+\mathbf{e}_j\mathbf{S})} \nonumber\\
			& \frac{-i^{\mathbf{e}_j \mathbf{S e}_j^T}}{N} \left(\sum_{\mathbf{a} \in \mathbb{F}^m_2}{(-1)^{\mathbf{a S }\mathbf{e}_j^T} \mathbf{D(a,0)}} \right) \mathbf{D(0},\mathbf{e}_j+\mathbf{e}_j \mathbf{S}) \label{FinalRes}
		\end{align}
		\\
		
		By considering expression inside the parentheses in  Eq. \eqref{Hrefomed} and multiplying it by $\mathbf{H}_N^{\text{nat}}$ from left, we have
		\begin{align}
			&\mathbf{H}_N^{\text{nat}} \sum_{\mathbf{a} \in \mathbb{F}^m_2}{(-1)^{\mathbf{a S }\mathbf{e}_j^T} \mathbf{D(a,0)}}= \frac{1}{\sqrt{N}} \sum_{\mathbf{u}, \mathbf{v} \in \mathbf{F}_2^m}{\left(-1\right)^{\mathbf{u} \mathbf{v}^T } |\mathbf{u}><\mathbf{v}|} \sum_{\mathbf{a}, \mathbf{x} \in \mathbf{F}_2^m}{\left(-1\right)^{\mathbf{S}_j \mathbf{a}^T } |\mathbf{x+a}><\mathbf{x}|} \nonumber \\
			& \frac{1}{\sqrt{N}} \sum_{\mathbf{u}, \mathbf{x}, \mathbf{a} \in \mathbf{F}_2^m}{(-1)^{(\mathbf{x+a})\mathbf{u}^T} (-1)^{\mathbf{S}_j \mathbf{a}^T}|\mathbf{u}><\mathbf{x}| }= \nonumber \\ 
			&\frac{1}{\sqrt{N}} \sum_{\mathbf{u}, \mathbf{x} \in \mathbf{F}_2^m}{(-1)^{\mathbf{u} \mathbf{x}^T} \sum_{ \mathbf{a} \in \mathbf{F}_2^m}{(-1)^{(\mathbf{u+S}_j)\mathbf{a}^T}} |\mathbf{u}><\mathbf{x}| } = \nonumber \\
			&\sqrt{N} \sum_{ \mathbf{x} \in \mathbf{F}_2^m}{(-1)^{\mathbf{S}_j \mathbf{x}^T} |\mathbf{S}_j><\mathbf{x}| }\label{HsubRes}
		\end{align}
		where the last equality come from the fact that 
		\begin{equation}
			\sum_{ \mathbf{a} \in \mathbf{F}_2^m}{(-1)^{(\mathbf{u+S}_j)\mathbf{a}^T}} = 
			\begin{cases}
				0 & \qquad \mathbf{u} \neq \mathbf{S}_j \\
				N & \qquad \mathbf{u} = \mathbf{S}_j
			\end{cases}
		\end{equation}
		
		
		We notice that $|\mathbf{S}_j>$ is a map between $j$th row of $\mathbf{S}$ and basic vectors in $2^m$. Actually $|\mathbf{S}_j>$ has a non-zero element at location of decimal mapping of $j$th row of $\mathbf{S}$. Thus finally considering \eqref{FinalRes} and \eqref{HsubRes}, we have
		\begin{align}
			\mathbf{H}_N^{\text{nat}}\overline{\mathbf{W}} \odot \mathbf{E}(\mathbf{e}_n, 0)\mathbf{W} = \frac{-1 (i)^{\mathbf{e}_j \mathbf{S e}_j^T}}{\sqrt{N}} \sum_{ \mathbf{x} \in \mathbf{F}_2^m}{(-1)^{\mathbf{e}_j \mathbf{x}^T} |\mathbf{S}_j><\mathbf{x}| }\label{FinHcalRes}
		\end{align}
	\item Second case: $\mathbf{W=gHg}^H$. Now calculate a "shift" of the codeword matrix using Eq. \ref{WsecDef}, We have
	\begin{align}
		\mathbf{E}(\mathbf{e}_j, \mathbf{0})\mathbf{W} = &\frac{1}{\sqrt{N}} \sum_{\mathbf{a} \in \mathbb{F}^m_2} i^{w(\mathbf{a})}\mathbf{E}(\mathbf{e}_j,\mathbf{0})\mathbf{E}(\mathbf{a},\mathbf{a}\mathbf{S} + \mathbf{a}) = \frac{1}{\sqrt{N}} \sum_{\mathbf{a} \in \mathbb{F}^m_2} i^{w(\mathbf{a})}i^{-\mathbf{e}_j(\mathbf{a}\mathbf{S} + \mathbf{a})^T}\mathbf{E}(\mathbf{a} + \mathbf{e}_j,\mathbf{a}\mathbf{S} + \mathbf{a}) \nonumber \\ 
		 &= \frac{1}{\sqrt{N}} \sum_{\mathbf{a} \in \mathbb{F}^m_2} i^{w(\mathbf{a})-\mathbf{a}(\mathbf{S} + \mathbf{I})\mathbf{e}_j^T}\mathbf{E}(\mathbf{a} + \mathbf{e}_j,\mathbf{a}\mathbf{S} + \mathbf{a}) \label{EjWSecDef}
	\end{align}
	
	
Using Eq. \ref{WbarSecDef} and \ref{EjWSecDef}, we have
	
	\begin{align*}
		\overline{\mathbf{W}} \odot \mathbf{E}(\mathbf{e}_j, 0)\mathbf{W} = \frac{1}{N} \sum_a i^{-w(\mathbf{a})}(-1)^{\mathbf{a}(\mathbf{a}\mathbf{S} + \mathbf{a})^T} E(\mathbf{a}, \mathbf{a}(\mathbf{S} + \mathbf{I})) \odot \sum_b i^{w(\mathbf{b}) - \mathbf{b}(\mathbf{S} + \mathbf{I})\mathbf{e}_j^T} E(\mathbf{b} + \mathbf{e}_j, \mathbf{b}(\mathbf{S} + \mathbf{I}))
	\end{align*}

	so this means that
	\begin{align}
		&=\frac{1}{N} \sum_{\mathbf{a}\in \mathbb{F}_2^m} (-1)^{\mathbf{a}(\mathbf{a}\mathbf{S} + \mathbf{a})^T} i^{-w(\mathbf{a})} E(\mathbf{a}, \mathbf{a}(\mathbf{S} + \mathbf{I})) \odot \sum_{\mathbf{b}\in \mathbb{F}_2^m} i^{w(\mathbf{b}) - \mathbf{b}(\mathbf{S} + \mathbf{I})\mathbf{e}_j^T} E(\mathbf{b} + \mathbf{e}_j, \mathbf{b}(\mathbf{S} + \mathbf{I})) \nonumber \\
		&=\frac{1}{N}\sum_{\mathbf{a},\mathbf{b}\in \mathbb{F}_2^m} \delta_{\mathbf{a}, \mathbf{b} + \mathbf{e}_j} (-1)^{w(\mathbf{b})-w(\mathbf{a})} i^{\mathbf{b S b}^T-\mathbf{a S a}^T} D(\mathbf{a}, \mathbf{a}(\mathbf{S}+\mathbf{I})) \odot D(\mathbf{b} + \mathbf{e}_j, \mathbf{b}(\mathbf{S}+\mathbf{I})) \nonumber \\
		&=\frac{1}{N}\sum_{\mathbf{b}\in \mathbb{F}_2^m}{(-1)^{w(\mathbf{b})-w(\mathbf{b}\oplus \mathbf{e}_j)} i^{\mathbf{b S b}^T-\mathbf{(\mathbf{b}\oplus \mathbf{e}_j) S }(\mathbf{b}\oplus \mathbf{e}_j)^T} } \times \nonumber\\
		&\quad D(\mathbf{b}+\mathbf{e}_j, \mathbf{b}(\mathbf{S}+\mathbf{I}) +\mathbf{e}_j (\mathbf{S}+\mathbf{I})) \odot D(\mathbf{b} + \mathbf{e}_j, \mathbf{b}(\mathbf{S}+\mathbf{I})) \nonumber \\
		&\stackrel{(a)}{=}\frac{-1}{N}\sum_{\mathbf{b}\in \mathbb{F}_2^m} i^{\mathbf{b S b}^T-\mathbf{(\mathbf{b}\oplus \mathbf{e}_j) S }(\mathbf{b}\oplus \mathbf{e}_j)^T} D(\mathbf{b} + \mathbf{e}_j, \mathbf{e}_j(\mathbf{S}+\mathbf{I}))   \nonumber \\
		&=\frac{-i^{-\mathbf{e}_j \mathbf{S e}_j}}{N}\sum_{\mathbf{b}\in \mathbb{F}_2^m} (-1)^{\mathbf{b Se}_j^T} D(\mathbf{b} + \mathbf{e}_j, \mathbf{e}_j(\mathbf{S}+\mathbf{I}))
	\end{align}
	Thus that will be similar to the case that, we considered $\mathbf{w=gH}$ (refer to Eq. \ref{FinalRes}) and hence, the same procedure as in Eq. \ref{HsubRes} can be considered by multiplication of Hadamard matrix from the left and we will get similar result to Eq. \ref{FinHcalRes}. 
	
	We notice about $(a)$, it comes from the fact that If $\mathbf{a} \neq \mathbf{b} \implies E(\mathbf{a}, \mathbf{c}) \odot E(\mathbf{b}, \mathbf{d}) = \mathbf{0}$. Why this is true? First we notice that 
	\begin{equation*}
		\left(\mathbf{A} \otimes \mathbf{B} \right) \odot \left(\mathbf{C} \otimes  \mathbf{D} \right) = \left(\mathbf{A} \odot \mathbf{C} \right) \otimes \left(\mathbf{B} \odot \mathbf{D}\right)
	\end{equation*}	
	Then we notice that $\mathbf{x}$ changes position of diagonal to anti-diagonal elements and thus when consider above mentioned point we have $\left(\mathbf{x}^{a_1} \mathbf{z}^{b_1}  \right) \odot \left(\mathbf{x}^{c_1} \mathbf{z}^{d_1}  \right)$ when $a_1 \neq c_1$ result will be $\odot$ of diagonal and anti-diagonal and hence will be a all zero matrix. Also, we notice that
	 $D(\mathbf{a, b+c}) \odot D(\mathbf{a, b})=D(\mathbf{a, c})$. How we can prove that? Here is some insight. By considering the definition, it is suffice to consider first element for example:
	\begin{equation*}
		\mathbf{x}^{a_1}\mathbf{z}^{b_1+c_1} \odot \mathbf{x}^{a_1}\mathbf{z}^{b_1}
	\end{equation*}
	we have 8 different cases here, by checking each of them and considering the fact that $\mathbf{x}\mathbf{z}\odot \mathbf{x}\mathbf{z} = \mathbf{x}$ and $\mathbf{x}\odot \mathbf{x}\mathbf{z} = \mathbf{x}\mathbf{z}$, we can conclude the result. What we can say about $E(.,.)$? We can perform as follows
	\begin{equation*}
		E(\mathbf{a,b\oplus c}) \odot E(\mathbf{a,b}) = i^{\mathbf{a}(\mathbf{b\oplus c})^T}i^{\mathbf{a}\mathbf{b}^T}D(\mathbf{a,b+c}) \odot D(\mathbf{a,b}) = i^{2(\mathbf{b+b}*\mathbf{c})\mathbf{a}^T} E(\mathbf{a,c})
	\end{equation*}
	where $*$ denote the element-wise product. Thus result is  $(-1)^{(\mathbf{b+b}*\mathbf{c})\mathbf{a}^T}E(\mathbf{a,c})$. Hence for getting rid of the element-wise multiplication, we simply replaced the $E(.,.)$ by $D(.,.)$.
	
	\end{enumerate}

\end{document}