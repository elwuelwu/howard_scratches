\usepackage[colorlinks=true,pdfstartview=FitV,linkcolor=black,citecolor=black,urlcolor=blue,plainpages=false]{hyperref}
%\newcommand{\href}[2]{#2}
%\usepackage[numbers,sort&compress]{natbib}
%\usepackage{hypernat}
\usepackage{pifont}
\usepackage{graphicx}
\usepackage{xcolor}
\usepackage{t1enc}
\usepackage{times}
\usepackage{epsf,epsfig,subfigure}
%\usepackage{wrapfig}
\usepackage{amsmath,amssymb}
\usepackage{amsthm}
\usepackage{caption}
\usepackage{array}
\usepackage{algorithm,algpseudocode}
%\usepackage{lipsum}
%\usepackage{colortbl}q
\usepackage{color}
%\usepackage{floatrow}
%\usepackage{booktabs}
%\usepackage{paralist}
\usepackage{fancyref}
%\usepackage{bm}
%\usepackage{multirow}
%\usepackage{amssymb}
\usepackage{subfigure}
%\usepackage{cases}
%\usepackage{empheq}
%\usepackage{cite}
%\usepackage[nottoc,numbib]{tocbibind}
%\usepackage{setspace}
%\usepackage{hyperref}
%\newcommand{\href}[2]{{\tt #1}}
\usepackage{eurosym}
\usepackage{relsize}

%\usepackage{relsize}
%\renewcommand\RSpercentTolerance{0}
\newcommand{\tran}{^{\mbox{\scriptsize T}}}
%\DeclareMathOperator{\rank}{rank}
%\DeclareMathOperator{\tr}{tr}
%\DeclareMathOperator{\diag}{diag}
\renewcommand{\L}{\mathsf{L}}
%\DeclareMathOperator*{\argmin}{arg\,min}
%\DeclareMathOperator*{\argmax}{arg\,max}
\DeclareMathOperator*{\maxi}{max}
\DeclareMathOperator*{\mini}{min}
\DeclareMathOperator*{\find}{find}
%\DeclareMathOperator*{\card}{card}
%\DeclareMathOperator*{\rank}{rank}
%\DeclareMathOperator*{\st}{subject~to}
\DeclareMathOperator*{\st}{s.t.}
\DeclareMathOperator*{\red}{red}
%\renewcommand{\baselinestretch}{0.9}
%\renewcommand{\baselineskip}{0.4ex}
%\renewcommand{\thesection}{\arabic{section}.}
%\renewcommand{\thesubsection}{\arabic{section}\Alph{subsection}}

% space-saving techniques
\usepackage{microtype} % makes the text more dense but also nicer looking
\usepackage{enumitem}
\setlist[itemize]{leftmargin=*, itemsep=.1em, topsep=0.5em} % makes itemization a bit more compact (no indents)
\usepackage[font={small,it}]{caption} % makes smaller and italics captions!
%\newfloatcommand{capbtabbox}{table}[][\FBwidth] % needed to have tables on the side of figures

% USEFUL COMMENT COMMANDS
\newcommand{\cs}[1]{{\textcolor{red}{\bf \em [cs: #1]}}}
\newcommand{\HEREc}{\begin{center}\bf \textcolor{red}{---- chris' edits end here ----} \end{center}}

% TWEAK LAYOUT (DO NOT CHANGE)
\setcounter{tocdepth}{2}
\def\tablestrut{\vrule height1.25em depth0pt width0pt}
\def\topfraction{.9}
\def\bottomfraction{.9}
\def\textfraction{.1}

%\textheight 25cm
%\textheight 9in
%\textwidth 7in
%\oddsidemargin 0in
%\evensidemargin 0in
%\topmargin -1cm
%\topmargin -.5in %\advance \topmargin-\baselineskip \headsep .5in
\marginparwidth .7in \marginparsep .15in

%\setlength{\textwidth}{178 mm}
%\setlength{\evensidemargin}{-8 mm}
\setlength{\textwidth}{165 mm}
\setlength{\evensidemargin}{-3 mm}

\setlength{\oddsidemargin}{\evensidemargin}
%\setlength{\textheight}{248 mm}
\setlength{\textheight}{245 mm}
\setlength{\topmargin}{-20 mm}
%\setlength{\headheight}{12pt}
%\setlength{\footskip}{5mm}


\floatsep 8pt plus 2pt minus 2pt
\textfloatsep 8pt plus 2pt minus 2pt

% change spacings before and after sections etc.
\makeatletter
\renewcommand\section{\@startsection {section}{1}{\z@}%
	{-2ex \@plus -1ex \@minus -.2ex}%
	{1.5ex \@plus.2ex}%
	{\normalfont\large\bfseries}}
\renewcommand\subsection{\@startsection{subsection}{2}{\z@}%
	{-2ex\@plus -1ex \@minus -.2ex}%
	{1ex \@plus .2ex}%
	{\normalfont\normalsize\bfseries}}
\renewcommand\subsubsection{\@startsection{subsubsection}{3}{\z@}%
	{1.5ex\@plus 1ex \@minus .2ex}%
	{.5ex \@plus .2ex}%
	{\normalfont\normalsize\bfseries}}
\setcounter{secnumdepth}{3}
\renewcommand{\paragraph}{%
	\@startsection{paragraph}{4}%
	{\z@}{1.2ex \@plus 1ex \@minus .2ex}{-1em}%
	{\normalfont\normalsize\bfseries}}
\makeatother

\let\OLDthebibliography\thebibliography
\renewcommand\thebibliography[1]{
	\OLDthebibliography{#1}
	\setlength{\parskip}{0pt}
	\setlength{\itemsep}{0pt plus 0.3ex}
}

\usepackage{tcolorbox}
\tcbuselibrary{theorems}

\newtcbtheorem[number within=section]{mytheo}{My Theorem}%
{colback=green!5,colframe=green!35!black,fonttitle=\bfseries}{th}

\newtcbtheorem[number within=section]{mytheo2}{Postulate}{colback=green!5,colframe=green!35!black,fonttitle=\bfseries}{th}